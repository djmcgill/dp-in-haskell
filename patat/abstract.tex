\begin{abstract}

This paper investigates the potential benefits offered by adopting a
declarative approach, as embodied by modern functional languages with mature
implementations, for prototyping algorithms for solving combinatorial
optimisation problems. For this class of problems, there are usually many
different options for the core algorithms, supporting data structures, and
other implementation aspects. Thus, tools allowing different alternatives to
be tried out quickly, focusing on the essence of the problem, and as
unencumbered as possible by implementation detail, would be very useful. As a
case study, we consider dynamic programming algorithms. These have many uses
in scheduling and timetabling; either directly, or as a component within a
other methods such as branch-and-cut or column generation. Our findings
suggest that off-the-shelf, leading functional languages indeed can offer a
range of compelling advantages in this particular problem domain, while
yielding a performance that is adequate for verifying and evaluating the
implemented algorithms as such.

% In such cases, it
% would be useful to be able to take existing methods, specifically functional
% programming methods, that support rapid prototyping, and tailor them to the
% kinds of algorithm classes used by the optimisation community. 
% 
% One such class
% of algorithms is dynamic programming, which can have many uses in scheduling
% and timetabling; either directly, or as a component within a other methods
% such as branch-and-cut or column generation. Here we report on initial
% investigations on potential advantages to using Functional Programming
% methods within such OR optimisation systems development.

\AJP{in the above abstract: 
need to fix the grammar; say something about the actual results?
}

\NHN{Have reworked the grammar. Enough about results?}

\keywords{Haskell \and C \and Java \and Functional Programming \and Dynamic Programming \and Language Comparison}
% \PACS{PACS code1 \and PACS code2 \and more}
% \subclass{MSC code1 \and MSC code2 \and more}
\end{abstract}

