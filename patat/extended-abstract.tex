%%%%%%%%%%%%%%%%%%%%%%% file template.tex %%%%%%%%%%%%%%%%%%%%%%%%%
%
% This is a general template file for the LaTeX package SVJour3
% for Springer journals.          Springer Heidelberg 2010/09/16
%
% Copy it to a new file with a new name and use it as the basis
% for your article. Delete % signs as needed.
%
% This template includes a few options for different layouts and
% content for various journals. Please consult a previous issue of
% your journal as needed.
%
%%%%%%%%%%%%%%%%%%%%%%%%%%%%%%%%%%%%%%%%%%%%%%%%%%%%%%%%%%%%%%%%%%%
%
% First comes an example EPS file -- just ignore it and
% proceed on the \documentclass line
% your LaTeX will extract the file if required
%
\RequirePackage{fix-cm}
%
%\documentclass{svjour3}                     % onecolumn (standard format)
%\documentclass[smallcondensed]{svjour3}     % onecolumn (ditto)
\documentclass[smallextended]{svjour3}       % onecolumn (second format)
%\documentclass[twocolumn]{svjour3}          % twocolumn
%
\smartqed  % flush right qed marks, e.g. at end of proof
%
\usepackage{graphicx}
%
\usepackage{mathptmx}      % use Times fonts if available on your TeX system
%
% insert here the call for the packages your document requires
%\usepackage{latexsym}
\usepackage{epstopdf}
\usepackage{float}
\floatstyle{boxed}
\restylefloat{table}
\restylefloat{figure}

\usepackage{listings}
\usepackage{hyperref}
% etc.
%
% please place your own definitions here and don't use \def but
% \newcommand{}{}

\long\def\ignore#1{}
\long\def\NOTE#1{{[[\bf{}NOTE: #1]]}}
\long\def\AJP#1{{[[\bf{}AJP: #1]]}}
\long\def\NHN#1{{[[\bf{}NHN: #1]]}}
\long\def\DXM#1{{[[\bf{}DXM: #1]]}}
% \let\NOTE=\ignore \let\AJP=\ignore \let\NHN=\ignore %% for final version

% Insert the name of "your journal" with
% \journalname{myjournal}
%
\begin{document}

\title{An Investigation Into the Use of Haskell for Dynamic Programming
\thanks{
This work was funded in part by EPSRC grant EP/F033613/1. 
% Grants or other notes
%about the article that should go on the front page should be
%placed here. General acknowledgements should be placed at the end of the article.
}
}
%\subtitle{Do you have a subtitle?\\ If so, write it here}

%\titlerunning{Short form of title}        % if too long for running head

\author{
	David McGillicuddy\and
	Andrew J. Parkes\and
	Henrik Nilsson
}

%\authorrunning{Short form of author list} % if too long for running head

\institute{D. McGillicuddy,  A. J. Parkes and H. Nilsson \at
             School of Computer Science\\
              University Of Nottingham \\
              \email{\{dxm, ajp, nhn\}@cs.nott.ac.uk}           %  \\
}

%\date{Received: date / Accepted: date}
\date{\today}
% The correct dates will be entered by the editor

\maketitle

\begin{abstract}

This paper investigates the potential benefits offered by adopting a
declarative approach, as embodied by modern functional languages with mature
implementations, for prototyping algorithms for solving combinatorial
optimisation problems. For this class of problems there are usually many
different options for the core algorithms, supporting data structures and
other implementation aspects. Thus tools that allow different alternatives to
be tried out quickly, focusing on the essence of the problem, and as
unencumbered as possible by implementation detail, would be very useful. As a
case study, we consider dynamic programming algorithms. These have many uses
in scheduling and timetabling; either directly, or as a component within a
other methods such as branch-and-cut or column generation. Our findings
suggest that off-the-shelf, leading functional languages indeed can offer a
range of compelling advantages in this particular problem domain, while
yielding a performance that is adequate for verifying and evaluating the
implemented algorithms as such.

% In such cases, it
% would be useful to be able to take existing methods, specifically functional
% programming methods, that support rapid prototyping, and tailor them to the
% kinds of algorithm classes used by the optimisation community. 
% 
% One such class
% of algorithms is dynamic programming, which can have many uses in scheduling
% and timetabling; either directly, or as a component within a other methods
% such as branch-and-cut or column generation. Here we report on initial
% investigations on potential advantages to using Functional Programming
% methods within such OR optimisation systems development.

\AJP{in the above abstract: 
need to fix the grammar; say something about the actual results?
}

\NHN{Have reworked the grammar. Enough about results?}

\keywords{Haskell \and C \and Java \and Functional Programming \and Dynamic Programming \and Language Comparison}
% \PACS{PACS code1 \and PACS code2 \and more}
% \subclass{MSC code1 \and MSC code2 \and more}
\end{abstract}



%\section{Introduction}
%\label{intro}
%Your text comes here. Separate text sections with
%\section{Section title}
%\label{sec:1}
%Text with citations \cite{RefB} and \cite{RefJ}.
%\subsection{Subsection title}
%\label{sec:2}
%as required. Don't forget to give each section
%and subsection a unique label (see Sect.~\ref{sec:1}).
%\paragraph{Paragraph headings} Use paragraph headings as needed.
%\begin{equation}
%a^2+b^2=c^2
%\end{equation}
%
%% For one-column wide figures use
%\begin{figure}
%% Use the relevant command to insert your figure file.
%% For example, with the graphicx package use
%  \includegraphics{example.eps}
%% figure caption is below the figure
%\caption{Please write your figure caption here}
%\label{fig:1}       % Give a unique label
%\end{figure}
%%
%% For two-column wide figures use
%\begin{figure*}
%% Use the relevant command to insert your figure file.
%% For example, with the graphicx package use
%  \includegraphics[width=0.75\textwidth]{example.eps}
%% figure caption is below the figure
%\caption{Please write your figure caption here}
%\label{fig:2}       % Give a unique label
%\end{figure*}
%%
%% For tables use
%\begin{table}
%% table caption is above the table
%\caption{Please write your table caption here}
%\label{tab:1}       % Give a unique label
%% For LaTeX tables use
%\begin{tabular}{lll}
%\hline\noalign{\smallskip}
%first & second & third  \\
%\noalign{\smallskip}\hline\noalign{\smallskip}
%number & number & number \\
%number & number & number \\
%\noalign{\smallskip}\hline
%\end{tabular}
%\end{table}

\section{Introduction}

Over the last decade the speed of computers has increased by many orders of
magnitude but the speed of the typical programmer has not. In many cases it is far more important to quickly produce correct and robust
code than to optimise code for performance, and as computers continue to
become more powerful, while humans will essentially remain the same this is
ultimately going to become the norm. We argue that prototyping new heuristics
and algorithms for combinatorial optimisation is one area where speed of
development of correct code is already more important than absolute
performance.

We use Haskell to implement standard dynamic programming algorithms, including
bounded and unbounded knapsack, and column generation. Then we
compare with implementations in Java and C in terms of speed, conciseness,
modularity, as well as ease of parallelisation, refactoring, debugging and
reasoning. To make the comparisons fair we keep the structure of the code
similar across languages, except when taking advantage of specific language
features (e.g., pointers, objects or laziness). The implementations are
idiomatic and representative of an 'average' user, without non-portable micro-optimisations. In particular, standard libraries are used throughout for
data structures, mathematics and floating-point arithmetic with as little as possible implemented by hand.

TODO: talk about reasoning about code


% \section{Abstraction Of Common Patterns}
% \section{Folds: an example of functional refactoring}
\section{An Illustrative Example}

In a recent high profile case \cite{Herndon13}, a spreadsheet bug caused
erroneous results from an economical analysis to be published, possibly
influencing European policy%
\footnote{\url{www.bbc.co.uk/news/magazine-22223190}}.
The error was partly caused by an indexing mistake that accidentally excluded
several countries from the analysis, an example of operational details causing
problems.

As an analogy, consider summing a collection of numbers. In a declarative
language like Haskell, the collection of numbers (whether an array, a list or
otherwise) is simply passed to the \emph{sum} function. Indexing and
element-wise operations take place behind the scenes, completely eliminating
indexing errors. By contrast, in most spreadsheets, the range of cells to
be summed must be manually selected (e.g., ``C3:C100'')%
\footnote{While named ranges do exist, they still have to be manually specified
which just pushes the problem elsewhere.}.
This is error-prone, especially if ranges later need to be modified to
accommodate additional data.

Let us consider how similar ideas might improve a combinatorial
optimisation algorithm. Solving the unbounded knapsack problem involves
finding the Greatest Common Divisor (gcd) of an array $\vec{W}$ of $n$
weights. The function $gcd$ (which takes two integers and returns the
largest integer dividing them both) is associative. Thus:
\[
gcd (\vec{W}_0, \ldots, \vec{W}_{n-1}) =
gcd(\vec{W}_0,gcd(\vec{W}_1, \ldots,gcd (\vec{W}_{n-2},\vec{W}_{n-1})\cdots))
\]
As we can see, $gcd$ is applied pairwise to the capacity and each weight,
reducing to a single integer after $n$ calls. Figure \ref{fig:gcds:java} shows
the algorithm implemented in Java. Iteration over the elements has been
abstracted into a \emph{for-each} loop, avoiding having to specify the bounds
of the loop and explicit indexing into the array at the cost of some
flexibility. The accumulator variable $gcd\_all$ is initialised to
$capacity$ and then $gcd$'d with each weight, updating the accumulator
variable with the result of $gcd$ for each new $\vec{W}_i$. The C version of
the algorithm is almost identical, except that the indexing in that case has
to be done explicitly, adding further operational details. We omit it for
reasons of space.

Figure \ref{fig:gcds:haskell}) shows the Haskell version. Here the idiom
of of reducing a list by a binary function and accumulator is captured
by the function \lstinline|foldr1|. There is thus no need for the programmer
to specify the range of the loop or how the accumulator should be updated,
reducing the number of places where mistakes might be made.

Furthermore, since the definition of $gcd$ contains the rule
`\lstinline|gcd 1 _ = 1|',
which states that $\forall x. gcd (1, x) = 1$, it can be said to be
\emph{short-circuiting}; i.e., if the first argument is equal to 1 then, due
to lazy evaluation, the second argument is not inspected and is ignored.
Therefore $gcds$ will automatically stop once a $1$ is encountered without any
change to the loop itself. Achieving the same optimisation in Java (or C) wold
require fusing the definition of $gcd$ with the loop code, which would break
modularity, hamper reuse, and render the code significantly less readable.

% \begin{figure}
% \lstinputlisting[language=c, firstline=5, lastline=11]{code/gcds.c}
% \caption{C99}
% \label{fig:gcds:c}
% \end{figure}

\begin{figure}
\lstinputlisting[language=java, firstline=3, lastline=9]{code/gcds.java}
\caption{Java 7}
\label{fig:gcds:java}
\end{figure}

\begin{figure}
\lstinputlisting[language=haskell, firstline=3, lastline=9]{code/gcds.hs}
\caption{Haskell}
\label{fig:gcds:haskell}
\end{figure}






\section{Results and Conclusions}

Our findings so far suggest that functional languages supported by mature
implementations indeed can speed up development by allowing implementations to
stay close to specifications, taking advantage of specific language features
such as laziness, and eliminating certain classes of errors, without incurring
a performance penalty that is unacceptable for prototypes. Our present
benchmark results (unbounded knapsack in C and Haskell) suggest that the C
code is not more than about four times faster that the Haskell version.

\textbf{TODO:} Graph of knapsack results here? gcc -O2 vs ghc-O2

\textbf{TODO:} Work on the phrasing of the conclusions below.

Even if not intending to move to FP, then techniques originatinng from 
FP, such as patterns for rules, and first class functions  along with 
techniques such as fold, are making their way into "standard" languages 
(Scala, F\#, etc) should be considered for use in OR prototypes and 
implementations.


%\begin{acknowledgements}
%If you'd like to thank anyone, place your comments here
%and remove the percent signs.
%\end{acknowledgements}

% BibTeX users please use one of
%\bibliographystyle{spbasic}      % basic style, author-year citations
\bibliographystyle{spmpsci}      % mathematics and physical sciences
%\bibliographystyle{spphys}       % APS-like style for physics
\bibliography{cites}   % name your BibTeX data base

% Non-BibTeX users please use
%\begin{thebibliography}{}
%
% and use \bibitem to create references. Consult the Instructions
% for authors for reference list style.
%
%\bibitem{RefJ}
% Format for Journal Reference
%Author, Article title, Journal, Volume, page numbers (year)
% Format for books
%\bibitem{RefB}
%Author, Book title, page numbers. Publisher, place (year)
% etc
%\end{thebibliography}

\end{document}
% end of file template.tex
