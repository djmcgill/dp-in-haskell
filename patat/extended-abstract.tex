%%%%%%%%%%%%%%%%%%%%%%% file template.tex %%%%%%%%%%%%%%%%%%%%%%%%%
%
% This is a general template file for the LaTeX package SVJour3
% for Springer journals.          Springer Heidelberg 2010/09/16
%
% Copy it to a new file with a new name and use it as the basis
% for your article. Delete % signs as needed.
%
% This template includes a few options for different layouts and
% content for various journals. Please consult a previous issue of
% your journal as needed.
%
%%%%%%%%%%%%%%%%%%%%%%%%%%%%%%%%%%%%%%%%%%%%%%%%%%%%%%%%%%%%%%%%%%%
%
% First comes an example EPS file -- just ignore it and
% proceed on the \documentclass line
% your LaTeX will extract the file if required
\begin{filecontents*}{example.eps}
%!PS-Adobe-3.0 EPSF-3.0
%%BoundingBox: 19 19 221 221
%%CreationDate: Mon Sep 29 1997
%%Creator: programmed by hand (JK)
%%EndComments
gsave
newpath
  20 20 moveto
  20 220 lineto
  220 220 lineto
  220 20 lineto
closepath
2 setlinewidth
gsave
  .4 setgray fill
grestore
stroke
grestore
\end{filecontents*}
%
\RequirePackage{fix-cm}
%
%\documentclass{svjour3}                     % onecolumn (standard format)
%\documentclass[smallcondensed]{svjour3}     % onecolumn (ditto)
\documentclass[smallextended]{svjour3}       % onecolumn (second format)
%\documentclass[twocolumn]{svjour3}          % twocolumn
%
\smartqed  % flush right qed marks, e.g. at end of proof
%
\usepackage{graphicx}
%
\usepackage{mathptmx}      % use Times fonts if available on your TeX system
%
% insert here the call for the packages your document requires
%\usepackage{latexsym}
\usepackage{epstopdf}
\usepackage{float}
\floatstyle{boxed}
\restylefloat{table}
\restylefloat{figure}

\usepackage{listings}
\usepackage{hyperref}
% etc.
%
% please place your own definitions here and don't use \def but
% \newcommand{}{}
%
% Insert the name of "your journal" with
% \journalname{myjournal}
%
\begin{document}

\title{An Investigation Into the Use of Haskell for Dynamic Programming
%\thanks{Grants or other notes
%about the article that should go on the front page should be
%placed here. General acknowledgements should be placed at the end of the article.}
}
%\subtitle{Do you have a subtitle?\\ If so, write it here}

%\titlerunning{Short form of title}        % if too long for running head

\author{
	David McGillicuddy \and
	Andrew J. Parkes \and
	Henrik Nilsson
}

%\authorrunning{Short form of author list} % if too long for running head

\institute{D. McGillicuddy \at
              University Of Nottingham \\
              \email{dxm@cs.nott.ac.uk}           %  \\
%             \emph{Present address:} of F. Author  %  if needed
           \and
           A. J. Parkes \at
	University Of Nottingham \\
	\email{ajp@cs.nott.ac.uk}
	\and
	H. Nilsson \at
	University Of Nottingham \\
	\email{nhn@cs.nott.ac.uk}
}

\date{Received: date / Accepted: date}
% The correct dates will be entered by the editor


\maketitle

\begin{abstract}
Over the last decade the speed of computers has increased by many orders of
magnitude but the speed of the typical programmer has not. In many cases it is far more important to quickly produce correct and robust
code than to optimise code for performance, and as computers continue to
become more powerful, while humans will essentially remain the same this is
ultimately going to become the norm. We argue that prototyping new heuristics
and algorithms for combinatorial optimisation is one area where speed of
development of correct code is already more important than absolute
performance.

As an example, there was a high profile case recently\cite{Herndon13} in which a
software bug has caused erroneous results to be published, and to possibly influence European policy\footnote{\url{www.bbc.co.uk/news/magazine-22223190}}. Amongst other things, a cause of error was a range-indexing mistake in the spreadsheet which caused several countries to be excluded from the analysis.

Consider the simplified example of summing a collection of numbers: in a high level programming language like Haskell one passes the name of a collection of numbers (whether an array, a list or otherwise) to the \emph{sum} function which will, behind the scenes and opaque to the user, index each element and add them together thus completely eliminating that class of errors. In most spreadsheet software however one has to manually select the cells (e.g. ``C3:C100'')\footnote{While named ranges do exist they still have to be manually specified which just pushes the problem elsewhere.} which is error-prone as well as being non-trivial to later expand to include additional data.

TODO: talk about reasoning about code

We use Haskell to implement standard dynamic programming algorithms, including
bounded and unbounded knapsack, and column generation. Then we
compare with implementations in Java and C in terms of speed, conciseness,
modularity, as well as ease of parallelisation, refactoring, debugging and
reasoning. To make the comparisons fair we keep the structure of the code
similar across languages, except when taking advantage of specific language
features (e.g., pointers, objects or laziness). The implementations are
idiomatic and representative of an 'average' user, without non-portable micro-optimisations. In particular, standard libraries are used throughout for
data structures, mathematics and floating-point arithmetic with as little as possible implemented by hand.

Our preliminary results (unbounded knapsack in C and Haskell) show that while
the C code is about four times faster, using Haskell for prototyping indeed
offers significant advantages in terms of speed of development and eliminating
certain classes of errors, without incurring a performance penalty that is
unacceptable for a prototype.

Consider the following examples from the unbounded knapsack implementations to find the Greatest Common Divisor (gcd) of an array of $n$ weights $\vec{W}$ and the initial capacity $c$. The function $gcd$ (which takes two integers and returns the largest number which divides both of them) is associative, so $$ gcd (c, \vec{W}_0, \ldots, \vec{W}_{n-1}) =  gcd(gcd(\cdots gcd(gcd(c,\vec{W}_0),\vec{W}_1)\cdots),\vec{W}_{n-1}) $$ and the code needs to apply $gcd$ pairwise to the capacity and each weight, reducing them to a single integer after $n$ calls to $gcd$. The basic algorithm is demonstracted in the C implementation\ref{fig:gcds:c} - there exists an accumulator variable $gcd_all$ which it initialised to $capacity$ and then $gcd$'d with each weight. Note that it was necessary to manually specify the bounds of the loop, index each element separately and then update the accumulator variable manually with the result of $gcd$ for each new $\vec{W}_i$.

Looping over each element of an array is such a common task that in Java it has been abstracted over with the \emph{for-each} loop. This avoids the problem of having to manually specify the bounds of the loop or indexing into the array at the cost of a lack of flexability - since the current index is not in scope as it would be inside a normal for loop, a for-each loop is not a good fit for code in which access to the index is needed. In this case however, as shown in the Java implementation of gcds\ref{fig:gcds:java}, that is not necessary so using a for-each loop reduces the boilerplate that the user needs to type and hence the number of places that an error can appear. The accumulator variable still needed to be manually updated in each iteration however.

In Haskell the idiom of updating an accumulator variable with each element of a list using a binary function is abstracted over using the function $foldl'$ defined by $$foldl' (f, acc0, \vec{W}) = f(f(\cdots f(f(acc0,\vec{W}_0),\vec{W}_1)\cdots),\vec{W}_{n-1})$$ and so the Haskell definition of gcds\ref{fig:gcds:haskell} calls $foldl'$ with the function $gcd$, the intial accumulator value of $c$ and the vector $\vec{W}$ thus avoiding the need for the programmer to manually specify the range of the loop or how the accumulator should be updated.


TODO: benchmarking gcd isn't the whole story due to fusion etc

\begin{figure}
\label{fig:gcds:C99}
\lstinputlisting[language=c, firstline=13, lastline=19]{code/gcds.c}
\caption{C99}
\end{figure}
\begin{figure}
\label{fig:gcds:java}
\lstinputlisting[language=java, firstline=3, lastline=9]{code/gcds.java}
\caption{Java 7}
\end{figure}
\begin{figure}
\label{fig:gcds:haskell}
\lstinputlisting[language=haskell, firstline=4, lastline=5]{code/gcds.hs}
\caption{Haskell}
\end{figure}



TODO: Graph of knapsack results here? gcc -O2 vs ghc-O2

\keywords{Haskell \and C \and Java \and Functional Programming \and Dynamic Programming \and Language Comparison}
% \PACS{PACS code1 \and PACS code2 \and more}
% \subclass{MSC code1 \and MSC code2 \and more}
\end{abstract}

%\section{Introduction}
%\label{intro}
%Your text comes here. Separate text sections with
%\section{Section title}
%\label{sec:1}
%Text with citations \cite{RefB} and \cite{RefJ}.
%\subsection{Subsection title}
%\label{sec:2}
%as required. Don't forget to give each section
%and subsection a unique label (see Sect.~\ref{sec:1}).
%\paragraph{Paragraph headings} Use paragraph headings as needed.
%\begin{equation}
%a^2+b^2=c^2
%\end{equation}
%
%% For one-column wide figures use
%\begin{figure}
%% Use the relevant command to insert your figure file.
%% For example, with the graphicx package use
%  \includegraphics{example.eps}
%% figure caption is below the figure
%\caption{Please write your figure caption here}
%\label{fig:1}       % Give a unique label
%\end{figure}
%%
%% For two-column wide figures use
%\begin{figure*}
%% Use the relevant command to insert your figure file.
%% For example, with the graphicx package use
%  \includegraphics[width=0.75\textwidth]{example.eps}
%% figure caption is below the figure
%\caption{Please write your figure caption here}
%\label{fig:2}       % Give a unique label
%\end{figure*}
%%
%% For tables use
%\begin{table}
%% table caption is above the table
%\caption{Please write your table caption here}
%\label{tab:1}       % Give a unique label
%% For LaTeX tables use
%\begin{tabular}{lll}
%\hline\noalign{\smallskip}
%first & second & third  \\
%\noalign{\smallskip}\hline\noalign{\smallskip}
%number & number & number \\
%number & number & number \\
%\noalign{\smallskip}\hline
%\end{tabular}
%\end{table}


%\begin{acknowledgements}
%If you'd like to thank anyone, place your comments here
%and remove the percent signs.
%\end{acknowledgements}

% BibTeX users please use one of
%\bibliographystyle{spbasic}      % basic style, author-year citations
\bibliographystyle{spmpsci}      % mathematics and physical sciences
%\bibliographystyle{spphys}       % APS-like style for physics
\bibliography{cites}   % name your BibTeX data base

% Non-BibTeX users please use
%\begin{thebibliography}{}
%
% and use \bibitem to create references. Consult the Instructions
% for authors for reference list style.
%
%\bibitem{RefJ}
% Format for Journal Reference
%Author, Article title, Journal, Volume, page numbers (year)
% Format for books
%\bibitem{RefB}
%Author, Book title, page numbers. Publisher, place (year)
% etc
%\end{thebibliography}

\end{document}
% end of file template.tex

