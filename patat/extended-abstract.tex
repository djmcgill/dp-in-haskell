%%%%%%%%%%%%%%%%%%%%%%% file template.tex %%%%%%%%%%%%%%%%%%%%%%%%%
%
% This is a general template file for the LaTeX package SVJour3
% for Springer journals.          Springer Heidelberg 2010/09/16
%
% Copy it to a new file with a new name and use it as the basis
% for your article. Delete % signs as needed.
%
% This template includes a few options for different layouts and
% content for various journals. Please consult a previous issue of
% your journal as needed.
%
%%%%%%%%%%%%%%%%%%%%%%%%%%%%%%%%%%%%%%%%%%%%%%%%%%%%%%%%%%%%%%%%%%%
%
% First comes an example EPS file -- just ignore it and
% proceed on the \documentclass line
% your LaTeX will extract the file if required
%
\RequirePackage{fix-cm}
%
%\documentclass{svjour3}                     % onecolumn (standard format)
%\documentclass[smallcondensed]{svjour3}     % onecolumn (ditto)
\documentclass[smallextended]{svjour3}       % onecolumn (second format)
%\documentclass[twocolumn]{svjour3}          % twocolumn
%
\smartqed  % flush right qed marks, e.g. at end of proof
%
\usepackage{graphicx}
%
\usepackage{mathptmx}      % use Times fonts if available on your TeX system
%
% insert here the call for the packages your document requires
%\usepackage{latexsym}
\usepackage{epstopdf}
\usepackage{float}
\floatstyle{boxed}
\restylefloat{table}
\restylefloat{figure}

\usepackage{listings}
\usepackage{hyperref}
% etc.
%
% please place your own definitions here and don't use \def but
% \newcommand{}{}

\long\def\ignore#1{}
\long\def\NOTE#1{{[[\bf{}NOTE: #1]]}}  
\long\def\AJP#1{{[[\bf{}AJP: #1]]}}  
\long\def\NHN#1{{[[\bf{}NHN: #1]]}}  
\long\def\DXM#1{{[[\bf{}DXM: #1]]}}  
% \let\NOTE=\ignore \let\AJP=\ignore \let\NHN=\ignore %% for final version

% Insert the name of "your journal" with
% \journalname{myjournal}
%
\begin{document}

\title{An Investigation Into the Use of Haskell for Dynamic Programming
%\thanks{Grants or other notes
%about the article that should go on the front page should be
%placed here. General acknowledgements should be placed at the end of the article.}
}
%\subtitle{Do you have a subtitle?\\ If so, write it here}

%\titlerunning{Short form of title}        % if too long for running head

\author{
	David McGillicuddy \and
	Andrew J. Parkes \and
	Henrik Nilsson
}

%\authorrunning{Short form of author list} % if too long for running head

\institute{D. McGillicuddy \at
             School of Computer Science\\
              University Of Nottingham \\
              \email{dxm@cs.nott.ac.uk}           %  \\
%             \emph{Present address:} of F. Author  %  if needed
           \and
           A. J. Parkes \at
             School of Computer Science\\
	University Of Nottingham \\
	\email{ajp@cs.nott.ac.uk}
	\and
	H. Nilsson \at
    School of Computer Science\\
	University Of Nottingham \\
	\email{nhn@cs.nott.ac.uk}
}

%\date{Received: date / Accepted: date}
\date{\today}
% The correct dates will be entered by the editor

\maketitle

\begin{abstract}

This paper investigates the potential benefits offered by adopting a
declarative approach, as embodied by modern functional languages with mature
implementations, for prototyping algorithms for solving combinatorial
optimisation problems. For this class of problems there are usually many
different options for the core algorithms, supporting data structures and
other implementation aspects. Thus tools that allow different alternatives to
be tried out quickly, focusing on the essence of the problem, and as
unencumbered as possible by implementation detail, would be very useful. As a
case study, we consider dynamic programming algorithms. These have many uses
in scheduling and timetabling: either directly, or as a component within
other methods such as column generation. Our findings
suggest that off-the-shelf, leading functional languages can indeed offer a
range of compelling advantages in this particular problem domain, while
yielding a performance that is adequate for verifying and evaluating the
implemented algorithms as such.

% In such cases, it
% would be useful to be able to take existing methods, specifically functional
% programming methods, that support rapid prototyping, and tailor them to the
% kinds of algorithm classes used by the optimisation community. 
% 
% One such class
% of algorithms is dynamic programming, which can have many uses in scheduling
% and timetabling; either directly, or as a component within a other methods
% such as branch-and-cut or column generation. Here we report on initial
% investigations on potential advantages to using Functional Programming
% methods within such OR optimisation systems development.

% \AJP{in the above abstract: 
% need to fix the grammar; say something about the actual results?
% }

% \NHN{Have reworked the grammar. Enough about results?}

\keywords{Haskell \and C \and Java \and Functional Programming \and Dynamic Programming \and Language Comparison}
% \PACS{PACS code1 \and PACS code2 \and more}
% \subclass{MSC code1 \and MSC code2 \and more}
\end{abstract}



%\section{Introduction}
%\label{intro}
%Your text comes here. Separate text sections with
%\section{Section title}
%\label{sec:1}
%Text with citations \cite{RefB} and \cite{RefJ}.
%\subsection{Subsection title}
%\label{sec:2}
%as required. Don't forget to give each section
%and subsection a unique label (see Sect.~\ref{sec:1}).
%\paragraph{Paragraph headings} Use paragraph headings as needed.
%\begin{equation}
%a^2+b^2=c^2
%\end{equation}
%
%% For one-column wide figures use
%\begin{figure}
%% Use the relevant command to insert your figure file.
%% For example, with the graphicx package use
%  \includegraphics{example.eps}
%% figure caption is below the figure
%\caption{Please write your figure caption here}
%\label{fig:1}       % Give a unique label
%\end{figure}
%%
%% For two-column wide figures use
%\begin{figure*}
%% Use the relevant command to insert your figure file.
%% For example, with the graphicx package use
%  \includegraphics[width=0.75\textwidth]{example.eps}
%% figure caption is below the figure
%\caption{Please write your figure caption here}
%\label{fig:2}       % Give a unique label
%\end{figure*}
%%
%% For tables use
%\begin{table}
%% table caption is above the table
%\caption{Please write your table caption here}
%\label{tab:1}       % Give a unique label
%% For LaTeX tables use
%\begin{tabular}{lll}
%\hline\noalign{\smallskip}
%first & second & third  \\
%\noalign{\smallskip}\hline\noalign{\smallskip}
%number & number & number \\
%number & number & number \\
%\noalign{\smallskip}\hline
%\end{tabular}
%\end{table}

\section{Introduction}

Over the last decade the speed of computers has increased by many orders of
magnitude but the speed of the typical programmer has not. In many cases it is
far more important to quickly produce correct and robust code than to optimise
code for performance, and as computers continue to become more powerful, while
humans will essentially remain the same, this is ultimately going to become
the norm. We argue that prototyping new heuristics and algorithms for
combinatorial optimisation is one area where speed of development of correct
code is already more important than absolute performance.

In this paper, we undertake a small case study as a preliminary investigation
into whether a declarative approach, specifically functional programming, is
feasible for this domain and if it indeed can help speeding up prototyping.
Our basic observation is that algorithms and heuristics for combinatorial
optimisation at their core have clear mathematical specifications, but that
implementation often is hampered by the need to spell out a plethora of
operational details which is time-consuming, error prone, and ultimately
obscure the essence of the code. Thus, if combinatorial optimisation
algorithms at least could be prototyped by, for the most part, transliterating
the mathematical specifications, and if the resulting performance were
adequate for evaluation purposes, much would be gained already. Additionally,
declaratively formulated code facilitates using techniques such as
property-based testing \cite{quickcheck}, which has proved highly effective
in many domains, and even to formally reason about code and prove aspects
of it correct.

For our case study, we have opted to look at a few standard dynamic
programming algorithms, including bounded and unbounded knapsack, as these
have many uses in scheduling and timetabling, and as they for our purposes are
representative of a larger class of algorithms in the domain of combinatorial
optimisation. For the declarative implementation framework, we have opted to
use the lazy functional language Haskell. This is because Haskell is a pure
language which increasess the contrast to the imperative languages commonly
used to implement this class of algorithms, making for a more interesting
comparision. Further, Haskell is supported by mature, industrial-strength
implementations, which makes for a fair comparision in terms of performance.


As a
case study, we consider dynamic programming algorithms. n. 



Haskell is one extreme

We use Haskell to implement standard dynamic programming algorithms, including
bounded and unbounded knapsack, and column generation. Then we compare with
implementations in Java and C in terms of speed, conciseness, modularity, as
well as ease of parallelisation, refactoring, debugging and reasoning. To make
the comparisons fair we keep the structure of the code similar across
languages, except when taking advantage of specific language features (e.g.,
pointers, objects or laziness). The implementations are idiomatic and
representative of an 'average' user, without non-portable micro-optimisations.
In particular, standard libraries are used throughout for data structures,
mathematics and floating-point arithmetic with as little as possible
implemented by hand.

Our aim is not to advocate any particular functional language.
Rather, it is about exploring what advantages functional notation
along with a mature implementation can bring to the table today,
out of the box, and which, if judged useful enough, with only
a little extra effort could be made available to domain-experts
without any particular background in FP, but who are interested
in expressing themselves declaratively, through FP-based DSLs
(with the mature FP implementations effectively being (part of)
the backend execution machinery).

One could strengthen that argument by citing some (E)DSL
success stories.

sophisticated off-the-shelf proprty-based testing tools like QuickCheck


TODO: talk about reasoning about code

Our findings
suggest that off-the-shelf, leading functional languages indeed can offer a
range of compelling advantages in this particular problem domain, while
yielding a performance that is adequate for verifying and evaluating the
implemented algorithms as such.


\section{Abstraction Of Common Patterns}
As an example, there was a high profile case recently\cite{Herndon13} in which a
software bug has caused erroneous results to be published, and to possibly influence European policy\footnote{\url{www.bbc.co.uk/news/magazine-22223190}}. Amongst other things, a cause of error was a range-indexing mistake in the spreadsheet which caused several countries to be excluded from the analysis.

Consider the simplified example of summing a collection of numbers: in a high level programming language like Haskell one passes the name of a collection of numbers (whether an array, a list or otherwise) to the \emph{sum} function which will, behind the scenes and opaque to the user, index each element and add them together thus completely eliminating that class of errors. In most spreadsheet software however one has to manually select the cells (e.g. ``C3:C100'')\footnote{While named ranges do exist they still have to be manually specified which just pushes the problem elsewhere.} which is error-prone as well as being non-trivial to later expand to include additional data.

Consider the following examples from the unbounded knapsack implementations to find the Greatest Common Divisor (gcd) of an array of $n$ weights, $\vec{W}$, and the initial capacity $c$. The function $gcd$ (which takes two integers and returns the largest integer which cleanly divides both of them) is associative, so $$ gcd (c, \vec{W}_0, \ldots, \vec{W}_{n-1}) =  gcd(gcd(\cdots gcd(gcd(c,\vec{W}_0),\vec{W}_1)\cdots),\vec{W}_{n-1}) $$ and the code needs to apply $gcd$ pairwise to the capacity and each weight, reducing them to a single integer after $n$ calls to $gcd$. The basic algorithm is demonstrated in the C implementation\ref{fig:gcds:c} - there exists an accumulator variable $gcd\_all$ which is initialised to $capacity$ and then $gcd$'d with each weight. Note that it was necessary to manually specify the bounds of the loop, index each element separately and then update the accumulator variable manually with the result of $gcd$ for each new $\vec{W}_i$.

In Java the task of iterating over each element has been abstracted into the \emph{for-each} loop. This avoids the problem of having to manually specify the bounds of the loop or index into the array at the cost of some flexibility. As shown in the Java implementation of gcds\ref{fig:gcds:java}, a for-each loop reduces the boilerplate that the user needs to type and hence the number of places that an error can appear.

In Haskell the idiom of updating an accumulator variable with each element of a list using a binary function is abstracted over using the function \lstinline|foldl'| as shown in the Haskell definition\ref{fig:gcds:haskell}, thus avoiding the need for the programmer to manually specify the range of the loop or how the accumulator should be updated and further reducing the places in which an error can appear.


TODO: make this way shorter

\begin{figure}
\label{fig:gcds:c}
\lstinputlisting[language=c, firstline=5, lastline=11]{code/gcds.c}
\caption{C99}
\end{figure}
\begin{figure}
\label{fig:gcds:java}
\lstinputlisting[language=java, firstline=3, lastline=9]{code/gcds.java}
\caption{Java 7}
\end{figure}
\begin{figure}
\label{fig:gcds:haskell}
\lstinputlisting[language=haskell, firstline=4, lastline=5]{code/gcds.hs}
\caption{Haskell}
\end{figure}






\section{Preliminary Results}
Our preliminary results (unbounded knapsack in C and Haskell) show that while
the C code is about four times faster, using Haskell for prototyping indeed
offers significant advantages in terms of speed of development and eliminating
certain classes of errors, without incurring a performance penalty that is
unacceptable for a prototype.


%\begin{acknowledgements}
%If you'd like to thank anyone, place your comments here
%and remove the percent signs.
%\end{acknowledgements}

% BibTeX users please use one of
%\bibliographystyle{spbasic}      % basic style, author-year citations
\bibliographystyle{spmpsci}      % mathematics and physical sciences
%\bibliographystyle{spphys}       % APS-like style for physics
\bibliography{cites}   % name your BibTeX data base

% Non-BibTeX users please use
%\begin{thebibliography}{}
%
% and use \bibitem to create references. Consult the Instructions
% for authors for reference list style.
%
%\bibitem{RefJ}
% Format for Journal Reference
%Author, Article title, Journal, Volume, page numbers (year)
% Format for books
%\bibitem{RefB}
%Author, Book title, page numbers. Publisher, place (year)
% etc
%\end{thebibliography}

\end{document}
% end of file template.tex
