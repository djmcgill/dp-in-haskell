% \section{Abstraction Of Common Patterns}
% \section{Folds: an example of functional refactoring}
\section{An Illustrative Example}
In a recent high profile case \cite{Herndon13}, a spreadsheet bug caused
erroneous results from an economical analysis to be published, possibly
influencing European Union policy%
\footnote{\url{www.bbc.co.uk/news/magazine-22223190}}.
The error was caused in part by an indexing mistake that accidentally excluded
several countries from the analysis, an example of operational details causing
problems.

As an analogy, consider summing a collection of numbers. In a declarative
setting
the numbers (whether in the form of an array, list, stream, or
otherwise) are simply passed to a generic \emph{sum} function. Indexing and
element-wise operations take place behind the scenes, completely eliminating
these as possible sources of programmer error. By contrast, in most spreadsheets, the range of cells to
be summed must be manually selected (e.g., ``C3:C100'')%
\footnote{While named ranges do exist, they still have to be manually specified
which just pushes the problem elsewhere.} which is error-prone.

Let us consider how similar ideas might improve a combinatorial
optimisation algorithm. Solving the unbounded knapsack problem involves
finding the Greatest Common Divisor (gcd) of an array $\vec{W}$ of $n$
weights and the initial capacity. The function $gcd$\footnote{Which takes two strictly positive integers and returns the largest integer that divides them both.}
is associative, thus it is applied pairwise to the capacity and each weight,
reducing to a single integer after $n$ calls. Figure \ref{fig:gcds:java} shows
the algorithm implemented in Java 7. Iteration over the elements has been
abstracted into a \emph{for-each} loop. The accumulator variable $gcd\_all$ is initialised to $capacity$ and then $gcd$'d with each weight, updating the accumulator
variable with the result of $gcd$ for each $\vec{W}_i$. The C version of
the algorithm (omitted for space reasons) is almost identical, except that the indices and loop ranges have to be written explicitly, adding further
operational details.

Figure \ref{fig:gcds:haskell} shows the Haskell version. Here the idiom
of reducing a list by a binary function and accumulator is captured
by the function \lstinline|foldr1|, so called because it folds,
associating to the right, over a list with at least one element. There is thus no 
need for the user to specify how and when the accumulator should be updated.

Furthermore, since the definition of $gcd$ contains the rule
`\lstinline|gcd 1 _ = 1|',
which states that $\forall x. gcd (1, x) = 1$, it can be said to be
\emph{short-circuiting}; i.e., if the first argument is equal to 1 then, due
to lazy evaluation, the second argument is not inspected and is ignored.
Therefore $gcds$ will automatically stop once a $1$ is encountered without any
change to the loop itself. Achieving the same optimisation in Java (or C) would
require changing the code for the loop itself, fusing into it some amount of the code
for $gcd$ itself, which would break
modularity, hamper reuse, and possibly rendering the code less readable. In this very small example, adding a check to see if $gcd\_all$ is equal to $1$ at each 
iteration, and halting the loop if so, is a trivial change but in larger programs, and ones in which the function inside the inner loop might not be so trivial or static,
this modification becomes less and less simple since the code that governs the loop might be
quite divorced from the code that updates the accumulating variable.

\begin{figure}
\lstinputlisting[language=java, firstline=3, lastline=9]{code/gcds.java}
\caption{Java 7}
\label{fig:gcds:java}
\end{figure}

%\begin{figure}
%\lstinputlisting[language=c, firstline=5, lastline=11]{code/gcds.c}
%\caption{C99}
%\label{fig:gcds:c}
%\end{figure}

\begin{figure}
\lstinputlisting[language=haskell, firstline=8, lastline=9]{code/gcds.hs}
\caption{Haskell}
\label{fig:gcds:haskell}
\end{figure}
