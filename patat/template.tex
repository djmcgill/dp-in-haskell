%%%%%%%%%%%%%%%%%%%%%%% file template.tex %%%%%%%%%%%%%%%%%%%%%%%%%
%
% This is a general template file for the LaTeX package SVJour3
% for Springer journals.          Springer Heidelberg 2010/09/16
%
% Copy it to a new file with a new name and use it as the basis
% for your article. Delete % signs as needed.
%
% This template includes a few options for different layouts and
% content for various journals. Please consult a previous issue of
% your journal as needed.
%
%%%%%%%%%%%%%%%%%%%%%%%%%%%%%%%%%%%%%%%%%%%%%%%%%%%%%%%%%%%%%%%%%%%
%
% First comes an example EPS file -- just ignore it and
% proceed on the \documentclass line
% your LaTeX will extract the file if required
\begin{filecontents*}{example.eps}
%!PS-Adobe-3.0 EPSF-3.0
%%BoundingBox: 19 19 221 221
%%CreationDate: Mon Sep 29 1997
%%Creator: programmed by hand (JK)
%%EndComments
gsave
newpath
  20 20 moveto
  20 220 lineto
  220 220 lineto
  220 20 lineto
closepath
2 setlinewidth
gsave
  .4 setgray fill
grestore
stroke
grestore
\end{filecontents*}
%
\RequirePackage{fix-cm}
%
%\documentclass{svjour3}                     % onecolumn (standard format)
%\documentclass[smallcondensed]{svjour3}     % onecolumn (ditto)
\documentclass[smallextended]{svjour3}       % onecolumn (second format)
%\documentclass[twocolumn]{svjour3}          % twocolumn
%
\smartqed  % flush right qed marks, e.g. at end of proof
%
\usepackage{graphicx}
%
\usepackage{mathptmx}      % use Times fonts if available on your TeX system
%
% insert here the call for the packages your document requires
%\usepackage{latexsym}
\usepackage{epstopdf}
% etc.
%
% please place your own definitions here and don't use \def but
% \newcommand{}{}
%
% Insert the name of "your journal" with
% \journalname{myjournal}
%
\begin{document}

\title{An Investigation Into the Use of Haskell for Dynamic Programming %\thanks{Grants or other notes
%about the article that should go on the front page should be
%placed here. General acknowledgements should be placed at the end of the article.}
}
%\subtitle{Do you have a subtitle?\\ If so, write it here}

%\titlerunning{Short form of title}        % if too long for running head

\author{David McGillicuddy \and
        Andrew J. Parkes \and
	Henrik Nilsson
}

%\authorrunning{Short form of author list} % if too long for running head

\institute{D. McGillicuddy \at
              University Of Nottingham \\
              \email{dxm@cs.nott.ac.uk}           %  \\
%             \emph{Present address:} of F. Author  %  if needed
           \and
           A. J. Parkes \at
	University Of Nottingham \\
	\email{ajp@cs.nott.ac.uk}
	\and
	H. Nilsson \at
	University Of Nottingham \\
	\email{nhn@cs.nott.ac.uk}
}

\date{Received: date / Accepted: date}
% The correct dates will be entered by the editor


\maketitle

\begin{abstract}
Over the last decades, the speed of computers has increased by many orders of
magnitude, but the speed of the typical programmer has not. In many cases it is far more important to quickly produce correct and robust
code than to optimise code for performance, and as computers continue to
become more powerful, while humans will essentially remain the same, this is
ultimately going to become the norm. We argue that prototyping new heuristics
and algorithms for combinatorial optimisation is one area where speed of
development of correct code is already more important than absolute
performance.

As an example, there was a high profile cases recently in which a
software bug has caused erroneous results to be published\cite{Herndon13}. % TODO: expand this

We use Haskell to implement standard dynamic programming algorithms, including
bounded and unbounded knapsack, and column generation. Then we
compare with implementations in Java and C in terms of speed, conciseness,
modularity, as well as ease of parallelisation, refactoring, debugging and
reasoning. To make the comparisons fair we keep the structure of the code
similar across languages, except when taking advantage of specific language
features (e.g., pointers, objects or laziness). The implementations are
idiomatic and representative of an 'average' user, without non- portable
micro-optimisations. In particular, standard libraries are used throughout for
floating-point arithmetic and non-trivial data structures.

Our preliminary results (unbounded knapsack in C and Haskell) show that while
the C code is about four times faster, using Haskell for prototyping indeed
offers significant advantages in terms of speed of development and eliminating
certain classes of errors, without incurring a performance penalty that is
unacceptable for a prototype.
\keywords{Haskell \and C \and Java \and Functional Programming \and Dynamic Programming \and Language Comparison}
% \PACS{PACS code1 \and PACS code2 \and more}
% \subclass{MSC code1 \and MSC code2 \and more}
\end{abstract}

%\section{Introduction}
%\label{intro}
%Your text comes here. Separate text sections with
%\section{Section title}
%\label{sec:1}
%Text with citations \cite{RefB} and \cite{RefJ}.
%\subsection{Subsection title}
%\label{sec:2}
%as required. Don't forget to give each section
%and subsection a unique label (see Sect.~\ref{sec:1}).
%\paragraph{Paragraph headings} Use paragraph headings as needed.
%\begin{equation}
%a^2+b^2=c^2
%\end{equation}
%
%% For one-column wide figures use
%\begin{figure}
%% Use the relevant command to insert your figure file.
%% For example, with the graphicx package use
%  \includegraphics{example.eps}
%% figure caption is below the figure
%\caption{Please write your figure caption here}
%\label{fig:1}       % Give a unique label
%\end{figure}
%%
%% For two-column wide figures use
%\begin{figure*}
%% Use the relevant command to insert your figure file.
%% For example, with the graphicx package use
%  \includegraphics[width=0.75\textwidth]{example.eps}
%% figure caption is below the figure
%\caption{Please write your figure caption here}
%\label{fig:2}       % Give a unique label
%\end{figure*}
%%
%% For tables use
%\begin{table}
%% table caption is above the table
%\caption{Please write your table caption here}
%\label{tab:1}       % Give a unique label
%% For LaTeX tables use
%\begin{tabular}{lll}
%\hline\noalign{\smallskip}
%first & second & third  \\
%\noalign{\smallskip}\hline\noalign{\smallskip}
%number & number & number \\
%number & number & number \\
%\noalign{\smallskip}\hline
%\end{tabular}
%\end{table}


%\begin{acknowledgements}
%If you'd like to thank anyone, place your comments here
%and remove the percent signs.
%\end{acknowledgements}

% BibTeX users please use one of
\bibliographystyle{spbasic}      % basic style, author-year citations
%\bibliographystyle{spmpsci}      % mathematics and physical sciences
%\bibliographystyle{spphys}       % APS-like style for physics
\bibliography{cites}   % name your BibTeX data base

% Non-BibTeX users please use
%\begin{thebibliography}{}
%
% and use \bibitem to create references. Consult the Instructions
% for authors for reference list style.
%
%\bibitem{RefJ}
% Format for Journal Reference
%Author, Article title, Journal, Volume, page numbers (year)
% Format for books
%\bibitem{RefB}
%Author, Book title, page numbers. Publisher, place (year)
% etc
%\end{thebibliography}

\end{document}
% end of file template.tex

