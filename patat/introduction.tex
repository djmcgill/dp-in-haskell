\section{Introduction}

Over the last few decades, the speed of computers has increased by orders of
magnitude, but the productivity of programmers has not kept pace. It is often far more
important to quickly produce correct and robust code than to optimise code for
performance, and as computers continue to become more powerful, while humans
will essentially remain the same, this is ultimately going to become the norm.
Prototyping new heuristics and algorithms for combinatorial optimisation is
arguably one area where speed of development of correct code is already more
important than absolute performance.

We undertake a small case study as a preliminary investigation
into whether a declarative approach, specifically functional programming, is
feasible for this domain and if it can help speed up prototyping. Our
basic observation is that algorithms and heuristics for combinatorial
optimisation at their core have clear mathematical specifications, but that
implementation often is hampered by the need to spell out a plethora of
operational details. This is time-consuming, error prone, and ultimately
obscures the essence of the code. Thus if combinatorial optimisation
algorithms could be prototyped by, for the most part, transliterating the
mathematical specifications, and if the resulting performance were adequate
for evaluation purposes, much would be gained already.
% Additionally, declaratively formulated code, as straightforward equational
% reasoning is valid, facilitates using techniques such as property-based
% testing \cite{quickcheck}, which has proved highly effective in many domains,
% and even formal reasoning and proving correctness.
Additionally because the elementary, ``school-book'' reasoning principle of
substituting equals for equals is valid in declaratively formulated code
(unlike in a setting with side effects), applying property-based testing
\cite{quickcheck} (where test cases are derived automatically from stated
correctness properties), more easily exploiting multi-core architectures, and formally proving correctness is potentially facilitated.

%We expect such issues to become even more important as optimisation methods move towards making effective usage of multi-core %machines with the associated challenges of debugging 

For our case study we have opted to look at a few standard dynamic
programming algorithms, specifically unbounded knapsack and longest common substring. These have
many uses in scheduling and timetabling and, for our purposes, are
representative of a larger class of algorithms in the domain of combinatorial
optimisation. We have opted to
use the pure, lazy, functional language Haskell \cite{Haskell98Book} as our declarative implementation framework. Using a pure language increases the contrast to the imperative languages commonly used
to implement this class of algorithms, making for a more interesting
comparison while also allowing the specific advantages of working
declaratively
% Could say: "as discussed above", but being less specific means this could
% also be taken to refer to other advantages such as being able to exploit
% laziness.
to be fully realised. Further, Haskell is supported by mature,
industrial-strength implementations, resulting in a fairer performance
comparison.

We would like to emphasise that our aim is not to advocate any particular
functional language for prototyping combinatorial optimisation algorithms.
Rather, we are interested in exploring what advantages functional notation
supported by mature implementations can bring today. However it is
worth noting that if these
advantages are judged to be compelling enough, functional
language implementations can, with relative ease, be leveraged for
implementing domain-specific languages (sometimes referred to as `executable specifications') that allow domain-experts interested in
working declaratively to reap the benefits of the approach without having to
become seasoned functional programmers themselves \cite{Hudak1998}. One example
of such a DSL, used to define and manipulate financial contracts,
was produced by Simon Peyton Jones et al. \cite{contracts} and
a derivative of it is used in industry by companies such as Bloomberg
and HSBC Private Bank.

We carry out the study by implementing each of the chosen algorithms (unbounded
knapsack and LCS) in
Haskell, Java, and C. The implementations are then compared along a number
of dimensions, including conciseness, modularity, performance, and ease of debugging, reasoning, and parallelising. To make the comparisons meaningful we retain the structure the
of the implementations across languages, except where we take advantage of
specific language features (such as pointers, objects, or laziness). The
implementations are further idiomatic and representative of what a
``typical'' programmer might write, without non-portable micro-optimisations.
In particular, standard libraries are used throughout for data structures and
mathematical computations, with as little as possible implemented from
scratch.

% Our findings so far suggest that functional languages supported by mature
% implementations indeed can speed up development by allowing implementations
% to stay close to specifications, taking advantage of specific language
% features such as laziness, and eliminating certain classes of errors,
% without incurring a performance penalty that is unacceptable for prototypes.

